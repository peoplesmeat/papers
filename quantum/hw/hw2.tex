\documentclass[11pt,fleqn]{article}
\usepackage{latexsym,epsf,epsfig}
\usepackage{amsmath,amsthm}
\usepackage{xy}
\input xy
\xyoption{all}
\begin{document}
\newcommand{\mbf}[1]{\mbox{{\bfseries #1}}}
\newcommand{\N}{\mbf{N}}
\renewcommand{\O}{\mbf{O}}
\newcommand{\ket}[1]{\left | #1 \right \rangle}
\newcommand{\bra}[1]{\left \langle #1 \right.|}
\newcommand{\braket}[2]{\left \langle #1 \right. \left | #2 \right \rangle}

\noindent Bill Davis \\
Homework 2 \\
October 21 2008

\begin{enumerate}
\item %Problem 1
$ \ket{\psi} = \frac{1}{\sqrt{3}}(\ket{0} \otimes \ket{0}) + 
\frac{2}{\sqrt{3}}(\ket{1} \otimes \ket{1}) = 
\left(\begin{array}{c}
\frac{1}{\sqrt{3}} \\
0 \\
0 \\
\frac{2}{\sqrt{3}} 
\end{array}\right)$

	
$ \ket{\psi} = \frac{1}{\sqrt{2}}(\ket{0} \otimes \ket{0}) + 
\frac{1}{\sqrt{2}}(\ket{1} \otimes \ket{1}) = 
\left(\begin{array}{c}
\frac{1}{\sqrt{2}} \\
0 \\
0 \\
\frac{1}{\sqrt{2}} 
\end{array}\right)$

Then $\braket{\phi}{\psi} = \frac{1}{\sqrt{3}}\frac{1}{\sqrt{2}} + \frac{2}{\sqrt{3}}\frac{1}{\sqrt{2}}$ 

\item %Problem 2
\begin{enumerate}
\item 

$\frac{1}{2}(\ket{00}+\ket{01} + \ket{10} + \ket{11}) = \frac{1}{\sqrt{2}}(\ket{0} + \ket{1}) \otimes \frac{1}{\sqrt{2}}(\ket{0} + \ket{1})$
\item
$\frac{1}{2}(\ket{00}+i\ket{01} + i\ket{10} - \ket{11}) = \frac{1}{\sqrt{2}}(\ket{0} + i\ket{1}) \otimes \frac{1}{\sqrt{2}}(\ket{0} + i\ket{1})$
\end{enumerate}

\item %Problem 3
For two arbitrary vectors \\
 $(a_{0}\ket{0} + a_{1}\ket{1}) \otimes (a_{2}\ket{0} + a_{3}\ket{1}) =$\\
$ 
a_{0}a_{2}\ket{00} + a_{0}a_{3}\ket{01} + a_{1}a_{2}\ket{01} + a_{1}a_{3}\ket{11}$


Therefore $ x_{0} = a_{0}a_{2} , x_1 = a_0a_3 , x_2=a_1a_2 , x_3=a_1a_3$ 

In this case $x_0x_3=a_0a_2a_1a_3 = x_1x_2$ If a value exists where $x_0x_3 \ne x_1x_2$ then this value cannot be expressed as a tensor product of two vectors. 
\item %Problem 4
\begin{enumerate}
\item 
$S_{x} \otimes S_{y}
\left(\left(\begin{array}{c} 
1\\ 
0\\ 
\end{array}\right) \otimes 
\left(\begin{array}{c} 
0\\ 
i\\ 
\end{array}\right)\right)=
\left(\begin{array}{c c c c} 
0 & 0 & 0 & -i\\ 
0 & 0 & i & 0\\ 
0 & -i & 0 & 0 \\
i & 0 & 0 & 0
\end{array}\right)
\left(\begin{array}{c} 
0\\ 
i\\ 
0\\
0 
\end{array}\right)=
\left(\begin{array}{c} 
0\\ 
0\\ 
1\\
0 
\end{array}\right)$
\item

$I \otimes S_{z}
\left(\left(\begin{array}{c} 
2\\ 
1\\ 
\end{array}\right) \otimes 
\left(\begin{array}{c} 
0\\ 
3\\ 
\end{array}\right)+ 
\left(\begin{array}{c} 
1\\ 
6\\ 
\end{array}\right)\otimes 
\left(\begin{array}{c} 
1\\ 
1\\ 
\end{array}\right)\right)=$ \\

$\left(\begin{array}{c c c c} 
1 & 0& 0& 0 \\
0 & -1 & 0 & 0 \\
0 & 0 & 1 & 0 \\
0 & 0 & 0 & -1
\end{array}\right)
\left(\begin{array}{c} 
1\\ 
7\\ 
6\\
4 
\end{array}\right)=
\left(\begin{array}{c} 
1\\ 
-7\\ 
6\\
4 
\end{array}\right)$

\end{enumerate}
\item %Problem 5
$\ket{\psi} = 
\left(\begin{array}{c} 
\frac{1}{\sqrt{2}} \\
0\\ 
0\\ 
0\\
0\\
0\\
0\\
\frac{1}{\sqrt{2}} 
\end{array}\right)
$

$\sigma_{y} \otimes \sigma_{y} \otimes \sigma_{x} = $ \\
$\left(\begin{array}{c c c c c c c c}
0 & 0 & 0 & 0 & 0 & 0 & 0 & -1\\
0 & 0 & 0 & 0 & 0 & 0 & -1 & 0 \\
0 & 0 & 0 & 0 & 0 & -1 & 0 & 0 \\
0 & 0 & 0 & 0 & -1 & 0 & 0 & 0 \\
0 & 0 & 0 & -1 & 0 & 0 & 0 & 0 \\
0 & 0 & -1 & 0 & 0 & 0 & 0 & 0 \\
0 & -1 & 0 & 0 & 0 & 0 & 0 & 0 \\
-1 & 0 & 0 & 0 & 0 & 0 & 0 & 0 
\end{array}\right)$

Then $\ket{\psi}$ is a eigenvector of $\sigma_{y} \otimes \sigma_{y} \otimes \sigma_{x}$ since $(\sigma_{y} \otimes \sigma_{y} \otimes \sigma_{x})\ket{\psi} = (-1)\ket{\psi}$ with eigenvalue -1. 

\item %Problem 6
 P($e_{1}) |\braket{e_{1}}{\psi}|^{2} = \frac{8}{25}$ \\
 P($e_{2})= |\braket{e_{2}}{\psi}|^{2} = \frac{9}{25}$ \\
 P($e_{2})= |\braket{e_{2}}{\psi}|^{2} = \frac{8}{25}$

\item %Problem 7
\begin{enumerate}
\item %Problem 7A
Let $A = \ket{\phi_1}\bra{\phi_1} + \ket{\phi_2}\bra{\phi_2} + \ket{\phi_3}\bra{\phi_3}$. In this case A can be represented by a 3x3 matrix.\\ We need to show that $A\ket{\psi} = I\ket{\psi} = \ket{\psi}$ for an arbitrary $\ket{\psi}$.

Since $(\phi_1, \phi_2, \phi_3)$ form an orthonormal basis we can express any arbitrary $\ket{\psi} = c_1\ket{\phi_1} + c_2\ket{\phi_2} + c_3\ket{\phi_3}$ where each coefficient $c_i = \braket{\phi_i}{\psi}$
\begin{eqnarray}
A\ket{\psi} &=&  (\ket{\phi_1}\bra{\phi_1} + \ket{\phi_2}\bra{\phi_2} + \ket{\phi_3}\bra{\phi_3})\psi \\
&=&\ket{\phi_1}\bra{\phi_1}\psi + \ket{\phi_2}\bra{\phi_2}\psi + \ket{\phi_3}\bra{\phi_3}\psi \\
&=&  c_1\ket{\phi_1} + c_2\ket{\phi_2} + c_3\ket{\phi_3} \\
&=& \ket{\psi}
\end{eqnarray}
Therefore, A=I.
\item %Problem 7B
$(\ket{\alpha}\bra{\beta}) =
\left(\begin{array}{c c c c }
\alpha_0 \\
\alpha_1 \\
\alpha_2 \\
...
\end{array}\right)
\left(\begin{array}{c c c c }
\beta_0^* & \beta_1^* & \beta_2^* & ... 
\end{array}\right)
=
\left(\begin{array}{c c c c }
\alpha_0\beta_0^{*} & \alpha_0\beta_1^{*} & \alpha_0\beta_2^{*} & ...\\
\alpha_1\beta_0^{*} & \alpha_1\beta_1^{*} & \alpha_1\beta_2^{*} & ...\\
\alpha_2\beta_0^{*} & \alpha_2\beta_1^{*} & \alpha_2\beta_2^{*} & ...\\
... & ... & ...
\end{array}\right)$

$(\ket{\beta}\bra{\alpha}) = 
\left(\begin{array}{c c c c }
\beta_0\alpha_0^{*} & \beta_0\alpha_1^{*} & \beta_0\alpha_2^{*} & ...\\
\beta_1\alpha_0^{*} & \beta_1\alpha_1^{*} & \beta_1\alpha_2^{*} & ...\\
\beta_2\alpha_0^{*} & \beta_2\alpha_1^{*} & \beta_2\alpha_2^{*} & ...\\
... & ... & ...
\end{array}\right)$

And because $A^{\dag}$ is the complex conjugate A with the rows and columns interchanged $(\ket{\alpha}\bra{\beta})^{\dag} = (\ket{\beta}\bra{\alpha})$.
\item %Problem 7C
If $A = \ket{\phi_1}\bra{\phi_2} + \ket{\phi_2}\bra{\phi_3} + \ket{\phi_3}\bra{\phi_1}$ then by ii $AA^{\dag} = $ 
\[
(\ket{\phi_1}\bra{\phi_2} + \ket{\phi_2}\bra{\phi_3} + \ket{\phi_3}\bra{\phi_1})
(\ket{\phi_2}\bra{\phi_1} + \ket{\phi_3}\bra{\phi_2} + \ket{\phi_1}\bra{\phi_3})
\]
And by i this equals $\ket{\phi_1}\bra{\phi_1} + \ket{\phi_2}\bra{\phi_2} + \ket{\phi_3}\bra{\phi_3} = I$

Therefore A is unitary. 
\end{enumerate}

\item %Problem 8
A and B are hermitian operators. Therefore their eigenvectors form an orthonormal basis. We need to show that if $\alpha_i$ are the eigenvectors of A and $\beta_i$ are the eigenvectors of B, that $\alpha_i=\beta_i$. 
\begin{eqnarray}
AB\alpha_i &=& BA\alpha_i \\
&=& Ba_i\alpha_i \\
&=& a_iB\alpha_i
\end{eqnarray}
There should be some way to show here that $B\alpha_i = Bb\alpha_i$, but I'm not seeing it.
\end{enumerate}


\end{document}
