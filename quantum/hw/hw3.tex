\documentclass[11pt,fleqn]{article}
\usepackage{latexsym,epsf,epsfig}
\usepackage{amsmath,amsthm}

\input {Qcircuit}
\xyoption{all}

\begin{document}
\newcommand{\mbf}[1]{\mbox{{\bfseries #1}}}
\newcommand{\N}{\mbf{N}}
\renewcommand{\O}{\mbf{O}}

\newcommand{\braket}[2]{\left \langle #1 \right. \left | #2 \right \rangle}

\noindent Bill Davis \\
Homework 3 \\
November 4 2008

\begin{enumerate}
\item %Problem 1
$\ket{\psi}= \frac{1}{2\sqrt{2}}((1+2i)\ket{\phi_1}+\sqrt{3}\ket{\phi_2})$ \\
$\ket{\xi}=\frac{1}{\sqrt{2}}(\ket{\phi_1}-i\ket{\phi_2})$

$\ket{\phi}\bra{\xi} = 
=\frac{1}{2\sqrt{2}}\begin{pmatrix}1+2i \\ \sqrt{3} \end{pmatrix} \frac{1}{\sqrt{2}} (1 , -i )^{*} = 
 \frac{1}{4}\left( \begin{array}{cc}
1+2i & -2+i  \\
\sqrt{3} & \sqrt{3}i  \end{array} \right)$

$\ket{\xi}\bra{\phi} = 
=\frac{1}{\sqrt{2}}\begin{pmatrix}1 \\ -i \end{pmatrix} \frac{1}{2\sqrt{2}} (1+2i ,2-i )^{*} = 
 \frac{1}{4}\left( \begin{array}{cc}
1-2i & \sqrt{3}  \\
-2-i & -\sqrt{3}i  \end{array} \right)$

\item %Problem 2

$\ket{\psi}=\frac{1}{\sqrt{3}}\ket{0} + \frac{\sqrt{2}}{\sqrt{3}}\ket{1}$

\begin{eqnarray}
HX\ket{\psi} &=& H\left( \begin{array}{cc}
						0 & 1  \\
						1 & 0  \end{array} \right)\ket{\psi} \\
&=& H\left( \begin{array}{cc}
						\frac{\sqrt{2}}{\sqrt{3}}   \\
						\frac{1}{\sqrt{3}}   \end{array} \right)\\
&=& \frac{1}{\sqrt{2}}\left( \begin{array}{cc}
						1 & 1   \\
						1 & -1  \end{array} \right)\left( \begin{array}{cc}
						\frac{\sqrt{2}}{\sqrt{3}}   \\
						\frac{1}{\sqrt{3}}   \end{array} \right) \\
&=& \frac{1}{\sqrt{2}} \left( \begin{array}{cc}
						\frac{\sqrt{2}}{\sqrt{3}}  +\frac{1}{\sqrt{3}} \\
						\frac{\sqrt{2}}{\sqrt{3}} -\frac{1}{\sqrt{3}}   \end{array} \right)
\end{eqnarray}

\item %Problem 3
$H\ket{0} = \frac{1}{\sqrt{2}}\left( \begin{array}{cc}
						1 & 1   \\
						1 & -1  \end{array} \right)\left( \begin{array}{cc}
						1   \\
						0   \end{array} \right) =  \frac{1}{\sqrt{2}}\left( \begin{array}{cc}
						1   \\
						1   \end{array} \right)$

$XH\ket{0} =X\frac{1}{\sqrt{2}}\left( \begin{array}{cc}
						1   \\
						1   \end{array} \right) = \frac{1}{\sqrt{2}}\left( \begin{array}{cc}
						1   \\
						1   \end{array} \right)$


CNOT($\frac{1}{\sqrt{2}}(\ket{0}+\ket{1}) \otimes \frac{1}{\sqrt{2}}(\ket{0}+\ket{1}))$

\item %Problem 4



\item %Problem 5
We can do this with three pieces of information.


\[ \Qcircuit @C=1em @R=.7em {
   & \gate{H} & \gate{H} & \qw 
} =
\Qcircuit @C=1em @R=.7em {
   & \qw & \qw
} \] 
\[
\frac{1}{\sqrt{2}}\left(\begin{array}{cc} 
1 & 1\\ 
1 & -1\\ 
\end{array}\right)
\frac{1}{\sqrt{2}}\left(\begin{array}{cc} 
1 & 1\\ 
1 & -1\\ 
\end{array}\right) = 
\frac{1}{2}\left(\begin{array}{cc} 
2 & 0\\ 
0 & 2\\ 
\end{array}\right) = I
\]
Second
\[
\Qcircuit @C=1em @R=.7em {
& \gate{X} &  \qw
} =
\Qcircuit @C=1em @R=.7em {
& \gate{H} & \gate{Z}  & \gate{H} &\qw
}
\]
\begin{eqnarray}
HZH &=&
\frac{1}{\sqrt{2}}\left(\begin{array}{cc} 
1 & 1\\ 
1 & -1\\ 
\end{array}\right)
\left(\begin{array}{cc} 
1 & 0\\ 
0 & -1\\ 
\end{array}\right)
\frac{1}{\sqrt{2}}\left(\begin{array}{cc} 
1 & 1\\ 
1 & -1\\ 
\end{array}\right) \\
&=& \frac{1}{2}
\left(\begin{array}{cc} 
1 & 1\\ 
1 & -1\\ 
\end{array}\right)
\left(\begin{array}{cc} 
1 & 1\\ 
-1 & 1\\ 
\end{array}\right) \\
&=& \frac{1}{2}
\left(\begin{array}{cc} 
0 & 2\\ 
2 & 0\\ 
\end{array}\right) \\
&=& 
\Qcircuit @C=1em @R=.7em {
 &\gate{X} & \qw 
}
\end{eqnarray}
This extends to comprobable controlled gates \\
Third 
\[
\Qcircuit @C=1em @R=.7em {
 &\ctrl{1}& \qw \\
 &\gate{Z}   & \qw
} = 
\Qcircuit @C=1em @R=.7em {
 &\gate{Z} & \qw\\
 &\ctrl{-1}& \qw 
}
\]
\[
\Qcircuit @C=1em @R=.7em {
 &\ctrl{1}& \qw \\
 &\gate{Z}   & \qw
} = 
\left(\begin{array}{cccc} 
1 & 0 & 0 & 0\\ 
0 & 1 & 0 & 0\\ 
0 & 0 & 1 & 0 \\
0 & 0 & 0 & -1
\end{array}\right) =
\Qcircuit @C=1em @R=.7em {
 &\gate{Z} & \qw\\
 &\ctrl{-1}& \qw 
}
\]

Then we can say that 
\begin{eqnarray}
\Qcircuit @C=1em @R=.7em {
 &\gate{H} & \ctrl{1} & \gate{H}& \qw\\
 &\gate{H} & \gate{X} & \gate{H}& \qw 
} &=& 
\Qcircuit @C=1em @R=.7em {
 &\gate{H} & \qw & \ctrl{1} & \qw & \gate{H}& \qw\\
 &\gate{H} &\gate{H} & \gate{Z}&\gate{H}  & \gate{H}& \qw 
} \\
&=&
\Qcircuit @C=1em @R=.7em {
 &\gate{H} & \qw & \ctrl{1} & \qw & \gate{H}& \qw\\
 & \qw & \qw &\gate{Z}& \qw &\qw &\qw 
}\\
&=&
\Qcircuit @C=1em @R=1em {
 &\gate{H} & \qw &\gate{Z}  & \qw & \gate{H}& \qw\\
 & \qw & \qw &\ctrl{-1} & \qw &\qw &\qw 
} \\
&=&
\Qcircuit @C=1em @R=1em {
 & \qw &\gate{X}  & \qw & \qw\\
 & \qw &\ctrl{-1} & \qw & \qw
}
\end{eqnarray}

\item
\begin{enumerate}
\item 


\begin{eqnarray}
(H \otimes H)(\ket{0}\otimes\ket{0}) &=&
\frac{1}{2}
\left(\begin{array}{cccc} 
1 & 1 & 1 & 1\\ 
1 & -1 & 1 & -1\\ 
1 & 1 & -1 & -1 \\
1 & -1 & -1 & 1
\end{array}\right)
\left(\begin{array}{c} 
1 \\
0 \\
0 \\
0 
\end{array}\right) \\
&=&
\frac{1}{2}
\left(\begin{array}{c} 
1 \\
1 \\
1 \\
1 
\end{array}\right)\\
&=&\frac{1}{2}(\ket{00} + \ket{01} + \ket{10} + \ket{11})
\end{eqnarray}

After applying the affect of $\frac{1}{2}U_f(\ket{00} + \ket{01} + \ket{10} + \ket{11})$ and using the notation in the handout equals 
\[
\frac{1}{2}(
\ket{0}\ket{f(0)}+\ket{0}\overline{\ket{f(0)}} +
\ket{1}\ket{f(1)}+\ket{1}\overline{\ket{f(1)}}
)
\]
Regardless of the output of the function f this can be reduced to 
\[
\frac{1}{2}(\ket{0}+\ket{1})(\ket{f(0)} + \overline{\ket{f(0)}} )
\]

Therefore the output of qbit 1 = $\frac{1}{\sqrt{2}}\ket{0}$ and for qbit 2 = $(\ket{f(0)} + \overline{\ket{f(0)}}0$
\item 
This setup does not enable to solve Deutsch's problem since we cannot descriminate based on the output of Qbit 1 whether the function is balenced or not since the output is the same regardless of $U_f$.
\end{enumerate}

\item %Problem 7

\begin{enumerate}
\item 
\[
\Qcircuit @C=1em @R=1em {
 & \ket{c} & \qw & \qw & \ctrl{2} & \qw \\
 & \ket{a} & \qw & \ctrl{1} & \qw & \qw\\
 & \ket{b} & \qw &\gate{X}  & \gate{X} & \qw
}
\]
\item 

\end{enumerate}
\end{enumerate}

\end{document}
