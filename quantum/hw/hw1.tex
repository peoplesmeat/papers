\documentclass[11pt,fleqn]{article}
\usepackage{latexsym,epsf,epsfig}
\usepackage{amsmath,amsthm}
\usepackage{xy}
\input xy
\xyoption{all}
\begin{document}
\newcommand{\mbf}[1]{\mbox{{\bfseries #1}}}
\newcommand{\N}{\mbf{N}}
\renewcommand{\O}{\mbf{O}}

\noindent Bill Davis \\
Homework 1 \\
October 7 2008


\begin{enumerate}
\item %Problem 1
\[ A= \left( \begin{array}{cc}
\frac{1+i}{\sqrt{3}} & \frac{1+i}{\sqrt{6}}  \\
\frac{i}{\sqrt{3}} & \frac{-2i}{\sqrt{6}}  \end{array} \right)\] 
A is unitary if $AA^{\dag}=I$
\[  
\left( \begin{array}{cc}
\frac{1+i}{\sqrt{3}} & \frac{1+i}{\sqrt{6}}  \\
\frac{i}{\sqrt{3}} & \frac{-2i}{\sqrt{6}}  \end{array} \right)
\left( \begin{array}{cc}
\frac{1-i}{\sqrt{3}} & \frac{-i}{\sqrt{3}}  \\
 \frac{1-i}{\sqrt{6}} & \frac{2i}{\sqrt{6}}  \end{array} \right)
 = 
\left( \begin{array}{cc}
1 & 0 \\
0 & 1 \end{array} \right)
\]
Therefore A is unitary.
\item %Problem 2
By the definition of a unitary operator
\[
(U \alpha, U \alpha) = (\alpha, \alpha) 
\] 
Since $\alpha$ is an eigenvector with value $c$. 
\[
(U \alpha, U \alpha) = (c \alpha, c \alpha)
\]
Therefore,
\[
(\alpha, \alpha)= (c \alpha, c \alpha) = c(c\alpha, \alpha)
 = c(\alpha, c\alpha)^{*} = cc^{*}(\alpha, \alpha)
\]
\[
(\alpha, \alpha) = cc^{*}(\alpha, \alpha)
\]
Then $1= cc^{*}$ and it follows that $|c|=1$.

\item %Problem 3
\begin{enumerate}
\item 
Assuming we have Hermitian operators $A_{1}$ and $A_{2}$ 
\[
	(\phi , (c_{1}A_{1}+c_{2}A_{2})\psi) = (\phi, c_{1}A_{1}\psi + c_{2}A_{2}\psi)  
\]
by the linearity of A. 
\begin{eqnarray}
(\phi, c_{1}A_{1}\psi + c_{2}A_{2}\psi) &=& c_{1}(\phi,A_{1}\psi) + c_{2}(\phi, A_{2}\psi) \\
&=& c_{1}(A_{1}\phi, \psi)+ c_{2}(A_{2}\phi, \psi) \\
&=& (c_{1}^{*}A_{1}\phi, \psi)+ (c_{2}^{*}A_{2}\phi, \psi) 
\end{eqnarray}
Where (1) is by I1 and (2) is by the definition of a hermitian operator. 
Because $c_{1}$ and $c_{2}$ are real $c_{1}=c_{1}^{*}$ and $c_{2}=c_{2}^{*}$
Therefore 
\[
(\phi , (c_{1}A_{1}+c_{2}A_{2})\psi) = ((c_{1}A_{1}+c_{2}A_{2})\phi, \psi)
\]
And $(c_{1}A_{1}+c_{2}A_{2})$ is hermitian

\item 
\begin{eqnarray}
(A_{1}A_{2}\phi, \psi) &=& (A_{1}(A_{2}\phi), \psi))\\
&=& (A_{2}\phi, A_{1}\psi)\\
&=& (\phi, A_{2}A_{1}\psi) 
\end{eqnarray}
And if $A_{1}A_{2}=A_{2}A_{1}$ then 
\[
(A_{1}A_{2}\phi, \psi) = (\phi, A_{1}A_{2}\psi)
\]
Therefore $A_{2}A_{1}$ is hermitian. 
\end{enumerate}
\item %Problem 4
Because A is observable, this means that A is hermitian. And by Thm 1.13 the eigenvectors of a Hermitian operator form a orthonormal basis for H. This means that $\psi$ can be expressed as 
\[
\psi = \sum_{m=1}^{M}{\psi_m}\left | a_{m} \right \rangle
\]
Also since we are certain that the measurement of observable A will equal $a_{k}$, we know that 
\begin{eqnarray}
P(a_{k})=1=| \left \langle a_{k} \right. \left | \psi \right \rangle |^{2} \\
P(a_{n\ne k})= 0 =| \left \langle a_{n\ne k} \right. \left | \psi \right \rangle |^{2}
\end{eqnarray}
And since the expansion coefficients of $\psi$ = $(a_{m},\psi)$ This means we can expand $\psi$ as $d_{1}\alpha_{1} + d_{2}\alpha_{2} + ... + d_{n} \alpha_{n}$ and by the above $d_{m} = 0$ for every value $m \ne k$. This means that $\psi = c\alpha_{k}$
\item %Problem 5

$  \left | \psi \right \rangle \left \langle \xi \right.| =\begin{pmatrix}9 \\ 2\end{pmatrix} (\frac{-i}{\sqrt{2}} , \frac{1}{\sqrt{2}})^{*} = 
 \left( \begin{array}{cc}
\frac{9i}{\sqrt{2}} & \frac{9}{\sqrt{2}}  \\
i\sqrt{2} &\sqrt{2}  \end{array} \right)$

$ \left | \xi \right \rangle \left \langle \psi \right.| =\begin{pmatrix}\frac{-i}{\sqrt{2}} \\ \frac{1}{\sqrt{2}} \end{pmatrix} (9 , 2 )^{*} = 
 \left( \begin{array}{cc}
\frac{-9i}{\sqrt{2}} & -i\sqrt{2}  \\
\frac{9}{\sqrt{2}} & \sqrt{2}  \end{array} \right)$
\item %Problem 6

\begin{enumerate}
\item The eigenvalues are $a_{1}=-1, a_{2}=0, a_{3}=1$
\item The probability of a measurement of A on a system in state $\psi$ at time t will yeild -1 is 
\[
| \left \langle a_{1} \right. \left | \psi \right \rangle |^{2} = (\frac{\sqrt{2}}{\sqrt{6}})^{2} = \frac{1}{3}
\]
\item
The probability of measuring $b_2$ given state $\psi$
\[
| \left \langle b_{2} \right. \left | \psi \right \rangle |^{2} = (\frac{2}{\sqrt{6}})^{2} 
\] 
And the probability of measuring $a_{3}$ given we have just measured $b_{2}$ is 
\[
| \left \langle  a_{3}  \right. \left |b_{2} \right  \rangle |^{2} = (\frac{1}{\sqrt{2}})^{2} 
\] 
So the probabaility of 0 then 1 = $(\frac{2}{3})(\frac{1}{2})=\frac{1}{3}$
\item
The probability of measuring $a_3$ and then $b_2$ is
\[ 
(\frac{1}{3})(\frac{1}{2})=\frac{1}{6}
\]
\end{enumerate}
\item %Problem 7
\begin{eqnarray}
(A\phi, \psi) &=& (A(d_{1}\alpha_{1}+d_{2}\alpha_{2} + ... + d_{n}\alpha_{n}), \psi) \\
&=& (a_{1}d_{1}\alpha_{1} + a_{2}d_{2}\alpha_{1} + ... + a_{2}d_{n}\alpha_{n}, \psi) \\
&=&(a_{1}d_{1}\alpha_{1} + ... + a_{2}d_{n}\alpha_{n} ,  
c_{1}\alpha_{1} + ... + c_{n}\alpha_{n})
\end{eqnarray}
For eigenvectors $\alpha_{n}$ and real eigenvalues $a_{n}$ And since $(\alpha_{n}, \alpha_{m})=0$ if $n\ne m$ and 1 otherwise then
\[
(A\phi, \psi) = a_{1}d_{1}c_{1}+ a_{2}d_{2}c_{2} + ... + a_{n}d_{n}c_{n}
\]

By the same argument as above  $(\phi, A\psi) = 
d_{1}a_{1}c_{1} + d_{2}a_{2}c_{2} + ... + d_{n}a_{n}c_{n}$
Since $a_{n}d_{n}c_{n}=d_{n}a_{n}c_{n}$ then A is hermitian. 
\end{enumerate}
\end{document}
