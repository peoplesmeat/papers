\documentclass[11pt,fleqn]{article}
\usepackage{latexsym,epsf,epsfig}
\usepackage{amsmath,amsthm}

\input {Qcircuit}
\xyoption{all}

\begin{document}
\newcommand{\mbf}[1]{\mbox{{\bfseries #1}}}
\newcommand{\N}{\mbf{N}}
\renewcommand{\O}{\mbf{O}}

\newcommand{\braket}[2]{\left \langle #1 \right. \left | #2 \right \rangle}

\noindent Bill Davis \\
Homework 5 \\
December 9th 2008


\begin{enumerate}
\item %Problem 1
\[
I \otimes H \otimes I = \frac{1}{\sqrt{2}}
 \left( \begin{array}{cccccccc}
						1 & 0 & 1  & 0 & 0 & 0 & 0 & 0    \\
						0 & 1 & 0  & 1 & 0 & 0 & 0 & 0    \\
						1 & 0 & -1 & 0 & 0 & 0 & 0 & 0    \\
                  0 & 1 & 0 & -1 & 0 & 0 & 0 & 0    \\
						0 & 0 & 0 & 0  & 1 & 0 & 1 & 0    \\
						0 & 0 & 0 & 0  & 0 & 1 & 0 & 0    \\
						0 & 0 & 0 & 0  & 1 & 0 & -1 & 0   \\
                  0 & 0 & 0 & 0  & 0 & 1 & 0 & -1    
						  \end{array} \right)
\]
\[
I \otimes CNOT = 
 \left( \begin{array}{cccccccc}
1 & 0 & 0 & 0 & 0 & 0 & 0 & 0 \\
0 & 1 & 0 & 0 & 0 & 0 & 0 & 0 \\
0 & 0 & 0 & 1 & 0 & 0 & 0 & 0 \\
0 & 0 & 1 & 0 & 0 & 0 & 0 & 0 \\
0 & 0 & 0 & 0 & 1 & 0 & 0 & 0 \\
0 & 0 & 0 & 0 & 0 & 1 & 0 & 0 \\
0 & 0 & 0 & 0 & 0 & 0 & 0 & 1 \\
0 & 0 & 0 & 0 & 0 & 0 & 1 & 0 \\
\end{array} \right)
\]
\[
CNOT \otimes I = 
 \left( \begin{array}{cccccccc}
1 & 0 & 0 & 0 & 0 & 0 & 0 & 0 \\
0 & 1 & 0 & 0 & 0 & 0 & 0 & 0\\
0 & 0 & 1 & 0 & 0 & 0 & 0 & 0 \\
0 & 0 & 0 & 1 & 0 & 0 & 0 & 0 \\
0 & 0 & 0 & 0 & 0 & 0 & 1 & 0 \\
0 & 0 & 0 & 0 & 0 & 0 & 0 & 1 \\
0 & 0 & 0 & 0 & 1 & 0 & 0 & 0 \\
0 & 0 & 0 & 0 & 0 & 1 & 0 & 0
\end{array} \right)
\]
\[
H \otimes I \otimes I = \frac{1}{\sqrt{2}}
 \left( \begin{array}{cccccccc}
1 & 0 & 0 & 0 & 1 & 0 & 0 & 0 \\
0 & 1 & 0 & 0 & 0 & 1 & 0 & 0 \\
0 & 0 & 1 & 0 & 0 & 0 & 1 & 0 \\
0 & 0 & 0 & 1 & 0 & 0 & 0 & 1 \\ 
1 & 0 & 0 & 0 & -1 & 0 & 0 & 0 \\
0 & 1 & 0 & 0 & 0 & -1 & 0 & 0 \\
0 & 0 & 1 & 0 & 0 & 0 & -1 & 0 \\
0 & 0 & 0 & 1 & 0 & 0 & 0 & -1
\end{array} \right)
\]

For the Toffoli gate $\ket{110} \rightarrow \ket{111}$ and $\ket{111} \rightarrow \ket{110}$
\[
T = 
 \left( \begin{array}{cccccccc}
1 & 0 & 0 & 0 & 0 & 0 & 0 & 0 \\
0 & 1 & 0 & 0 & 0 & 0 & 0 & 0 \\
0 & 0 & 1 & 0 & 0 & 0 & 0 & 0 \\
0 & 0 & 0 & 0 & 0 & 0 & 0 & 1 \\
0 & 0 & 0 & 0 & 1 & 0 & 0 & 0 \\
0 & 0 & 0 & 0 & 0 & 1 & 0 & 0 \\
0 & 0 & 0 & 0 & 0 & 0 & 1 & 0 \\
0 & 0 & 0 & 1 & 0 & 0 & 0 & 0 
\end{array} \right)
\]

\[
(I \otimes H \otimes I)
 \left( \begin{array}{cccccccc}
\alpha \\
0 \\ 
0 \\
0 \\
\beta \\
0 \\
0 \\
0 \\
\end{array} \right) = 
\frac{1}{\sqrt{2}}
 \left( \begin{array}{cccccccc}
\alpha \\
0 \\ 
\alpha \\
0 \\
\beta \\
0 \\
\beta \\
0 \\
\end{array} \right)
\] 
\[
(I  \otimes CNOT)\frac{1}{\sqrt{2}}
 \left( \begin{array}{cccccccc}
\alpha \\
0 \\ 
\alpha \\
0 \\
\beta \\
0 \\
\beta \\
0 \\
\end{array} \right) = \frac{1}{\sqrt{2}}
 \left( \begin{array}{cccccccc}
\alpha \\
0 \\
0 \\
\alpha \\
\beta \\
0 \\
0 \\
\beta
\end{array} \right)
\]
\[
CNOT \otimes I \frac{1}{\sqrt{2}}
 \left( \begin{array}{cccccccc}
\alpha \\
0 \\
0 \\
\alpha \\
\beta \\
0 \\
0 \\
\beta
\end{array} \right)= \frac{1}{\sqrt{2}}
 \left( \begin{array}{cccccccc}
\alpha \\
0 \\
0 \\
\alpha \\
0  \\
\beta \\
\beta \\
0 \\ 
\end{array} \right)
\]
\[
H \otimes I \otimes I \frac{1}{\sqrt{2}}
 \left( \begin{array}{cccccccc}
\alpha \\
0 \\
0 \\
\alpha \\
0  \\
\beta \\
\beta \\
0 \\ 
\end{array} \right) = \frac{1}{\sqrt{2}}
 \left( \begin{array}{cccccccc}
\alpha \\
\beta \\
\beta \\
\alpha \\
\alpha \\
-\beta \\
-\beta \\
\alpha \\ 
\end{array} \right) 
\]
\[
T  \frac{1}{\sqrt{2}}
 \left( \begin{array}{cccccccc}
\alpha \\
\beta \\
\beta \\
\alpha \\
\alpha \\
-\beta \\
-\beta \\
\alpha \\ 
\end{array} \right)  = \frac{1}{\sqrt{2}}
 \left( \begin{array}{cccccccc}
\alpha \\
\beta \\
\beta \\
\alpha \\
\alpha \\
-\beta \\
-\beta \\
\alpha \\ 
\end{array} \right) 
\]


\item %Problem 2
From (1) we know that pairs of particles exist with instruction set (A,1;B,?) where ? can be either a -1 or a 1.

 From (2) we know that there are no pairs of particles (A,1;B,?), (A,1;B,1) and from (3) we know that there are no particles (A,1;B,1), (A,1;B,?). Therefore there must be particles being generated with instruction sets (A,1;B,-1) for both particle 1 and particle 2.

 But (4) says that these particles do not exist therefore there is a contradiction. 
\item %Problem 3
\begin{enumerate}
\item 

Measuring a + for the first particle puts the two particle system into the state 
\[
\ket{+} \otimes \sqrt{\frac{3}{5}}\ket{-}-\sqrt{\frac{2}{5}}\ket{+}
\]
where $Prob(Measure \ket{+})$ for particle 1 = $\frac{3}{8}+\frac{1}{4}=\frac{5}{8}$

Measuring a - for the first particle puts the two particle system into the state 
\[
\ket{-}\ket{+}
\]
where $Prob(Measure \ket{+})$ for particle 1 = $\frac{3}{8}$

Using this information to complete the table

\begin{tabular}{ c c c }
 & +,+ & -,- \\
\hline
z,u & 0 & $\frac{3}{8}*\frac{2}{5} = \frac{3}{20}$\\
u,z & 0 & 0.15 \\
u, u & 0.09 & 0.64 \\
z,z & $\frac{5}{8} * \frac{2}{5}$ = $\frac{1}{4}$ & $\frac{3}{8}*0=0$\\
\hline
\end{tabular}

\item 
Here if we equate Z with measurement B and U with measurement A then this chart is identical to the equations 1-4. 

\end{enumerate}

\item %Problem 4
Since U is unitary,
\begin{eqnarray}
(U\ket{\psi}\ket{\alpha}, U\ket{\phi}\ket{\alpha}) &=& (\ket{\psi}\ket{\alpha}, \ket{\phi}\ket{\alpha}) \\
&=& (\ket{\psi}\ket{\beta}, \ket{\phi}\ket{\beta})
\end{eqnarray}

Then by the definition of the inner product over the tensor product,

\begin{eqnarray}
(\ket{\psi}\ket{\beta}, \ket{\phi}\ket{\beta}) &=&  (\ket{\psi}\ket{\alpha}, \ket{\phi}\ket{\alpha}) \\
\braket{\psi}{\phi}\braket{\beta}{\beta'} &=& \braket{\psi}{\phi}\braket{\alpha}{\alpha}
\end{eqnarray}

Because $\psi$ and $\phi$ are non-orthogonal, then $\braket{\psi}{\phi} \ne 0$
Therefore, 
\[
\braket{\alpha}{\alpha} = \braket{\beta}{\beta'}
\]
Where because $\ket{a}$ is some standard initial state $\braket{\alpha}{\alpha}=1$.
\[
\braket{\alpha}{\alpha} = 1 = \braket{\beta}{\beta'}
\]




\end{enumerate}
\end{document}
