\documentclass[11pt,fleqn]{article}
\usepackage{latexsym}
\usepackage{amsmath,amsthm}
\usepackage{xy}
\input xy
\xyoption{all}
\usepackage{url}
\begin{document}
\newcommand{\mbf}[1]{\mbox{{\bfseries #1}}}
\newcommand{\N}{\mbf{N}}
\renewcommand{\O}{\mbf{O}}

\noindent Bill Davis \\
Homework 8 \\
April 20 2011

\begin{document}

\begin{enumerate}

\item
Show that a one-time pad is perfect secrecy. Given a ciphertext c or length
|c| = n, we need to show that for any plaintext x the probability $P(E_k(x)=c) =
\frac{1}{2^n}$ In other words, every plaintext is as likely to generate c
as any other. First note that we can construct a key k, which when given input
x, $E_k(c)$ will be any arbitrary string of length n. This follows from the
definition of the xor operation. And since we randomly selected the one-time
pad, each key has probabality $\frac{1}{2^n}$ of being selected. Since each key
has equal probabality of being chosen, the probabality that the plaintext x
generated the ciphertext c is again $\frac{1}{2^n}$, in other words we gain no
information about x by examining c. 

\item
Show that when using an encryption scheme involving a key, if they key length is
less then the length of the message then the scheme cannot be perfectly secret.
Given a plaintext x, we can compute $E_k(x)=c_k$ for all possible keys k. By the
pigeonhole principal, there must be at least one value in the set of strings of
length $|c_k|$ which will not appear in the set of encrypted outputs. In other
words ${1,n}^* - E_k(x)$ is not empty, and if we ever received one of these
strings we would instantly know that it could not have been generated by the
plaintext x, regardless of the key. This would break perfect secrecy since we
could have extracted some information about x from c. 
  
\item
If P=NP then one-way functions cannot exist. One-way functions are
defined by the polynomial time computatable functions f. We can use f as a
verifier to construct an language in NP. If P=NP, then if there exists a
polynomial time validator, there also exists a polynomial time algorithm to
compute the inverse of f. In this case one-way functions cannot exist. 

   \item 
   Assume we have a one-way permutation f. We want to show that f^k is also a
   one-way permutation. For sake of contradiction assume not, meaning that f is
   a one-way permutation but f^k is not. Assume we have a bit string x, if we
   compute f(x), this is not invertible by a polynomial time algorithm by the
   definition of one-way. However if we compute $f^{k-1}(f(x))$, which is
   a polynomial computation in k-1 steps, we can then use a polynomial A to
   compute x since f^k is not one-way, but this contradicts the assumption that
   f is one-way.
   
\end{enumerate}

\end{document}